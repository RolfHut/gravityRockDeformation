\documentclass[a4paper,10pt,twoside,openany]{report}
\usepackage{graphicx}
%\usepackage{fancyhdr}
\usepackage{xcolor}
\usepackage{ifpdf}
\usepackage{ifthen}
\usepackage[round]{natbib}
\usepackage{amsmath}
\usepackage{units}
\usepackage[switch]{lineno}
\usepackage{xspace}






%%set font
\usepackage[T1]{fontenc}
%and allow UTF8 input ( e.g. umlauts and friends)
\usepackage[utf8]{inputenc}
\usepackage{palatino}
\usepackage{mathpazo}
%\usepackage{utopia}
%\usepackage{euler}


%define some colors
\definecolor{gfzblue}{rgb}{0., 0.4, 0.65}
\definecolor{iggblue}{rgb}{0,0.258824,0.568627}
%or gray
\definecolor{graycust}{rgb}{0.45,0.45,0.45}

\definecolor{backgrnd}{rgb}{0.8, 0.8, 0.8}
\definecolor{sunsetorange}{rgb}{1, .3, .11}
\definecolor{mossgreen}{rgb}{0.14,0.5,0.06}

%% Search path for figures
\graphicspath{{figures/}} %% full color figures
%%\graphicspath{{figures_gray/}} %% for grayscale figures

%%


%new commands to insert the title subtitle and author where necessary
\newcommand{\ititle}{Surface Loading effects on the LHC tunnel}
\title{\ititle}
\newcommand{\iauthor}{Roelof Rietbroek}
\author{\iauthor}

%add command to quickly insert colored remarks in the margin
\newcommand{\todo}[1]{\marginline{\color{red} TODO:\\#1}}

%pdf vs non-pdf options
\ifpdf
   \usepackage[pdftex,pdfpagelabels]{hyperref}
   \pdfcompresslevel=9
   \hypersetup{
     pdftitle={\ititle},
     pdfauthor={\iauthor},
     pdfsubject={LHC tunnel deformtions},
     pdfkeywords={Geodesy},
     colorlinks=true,
     urlcolor=iggblue,
     linkcolor=iggblue,
     citecolor=iggblue,
     breaklinks=true,
     plainpages=false
   }

\else
  \usepackage{hyperref} %basic dvi output with fugly  but obvious hyperlinks
     \hypersetup{
     colorlinks=true,
     urlcolor=iggblue,
     linkcolor=iggblue,
     citecolor=iggblue,
     breaklinks=true,
     plainpages=false
     }                           
  
\fi


%includeonly
%\includeonly{preface}
% begin main document
\begin{document}
\maketitle

\section{Introduction}
Some thoughts and a theoretical framework are gathered here to
quantify the circumference changes due to surface loading induced
deformations of the bedrock.

\section{Horizontal deformations due to surface loading}
\section{LHC circumference changes due to horizontal surface
  deformations}
Assumptions:
\begin{itemize}
\item ring is positioned horizontally (it is not but to first order it
  is)
  \item Horizontal deformations in and around the ring can be
    linearized wrt to the center point.
\end{itemize}
Circumference changes of the(a) LHC ring,  $\Delta L$, may be computed by a path
integral of the horizontal deformations, $\vec{h}$ along a circular path: 
\begin{equation}
  \Delta L =\oint_{ring} \vec{h} \cdot \frac{d\vec{s}}{|d\vec{s}|}
\end{equation}
For a circular ring we may parameterize the ring in terms of a fixed
radius, $r_{lhc}$ and an azimuth angle $\alpha$:
\begin{equation}
  \Delta L =\int_{0}^{2\pi}
  \left[\begin{array}{c}h_{north}(\alpha,r_{LHC})\\h_{east}(\alpha,r_{LHC})\end{array}\right]
  \cdot \left[\begin{array}{c}-\sin \alpha\\\cos \alpha\end{array}\right]d\alpha
\end{equation}


When the  ring is relatively small we may linearize the deformation
\subsection{Hydrology and atmospheric induced LHC circumference changes from GRACE data}
\subsection{LHC circumference changes due to water level changes}



% make Bibliography
%% \bibliography{roelofsrefsauto}
%% \bibliographystyle{abbrvnat}
%\bibliographystyle{plainnat}
%\bibliographystyle{authordate1}
\end{document}

